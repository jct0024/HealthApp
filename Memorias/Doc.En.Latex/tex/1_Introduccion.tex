\capitulo{1}{Introducción}

Actualmente se estima que alrededor del 81 por ciento de los españoles, que se proponen cambiar de hábitos y seguir una dieta acaba desistiendo \cite{81abandona}. Eso en su mayoría nace de la dificultad de mantener un ritmo constante de rigor a la hora de seguir una serie de instrucciones, que de una manera u otra te condicionan impidiéndote hacer una vida cien por cien normal, más allá de lo que en el ámbito alimenticio se refiere, como es en el ámbito social. \\

Por ello, combinando el análisis y tratamiento de datos mediante sistemas informáticos con los últimos estudios relacionados con la dietoterapia, se desarrolla este proyecto. No inculca una dieta estricta al usuario, sino que sirve de guía para percibir un nuevo estilo de vida como propio. Gracias a esto, se puede llegar a conseguir un gran avance médico/informático, debido a la importancia que la dietoterapia tiene en la salud de las personas. \\

El estudio \textit{Global Burden of Disease} en \textit{The Lancet}, que ha analizado las tendencias de consumo de quince factores dietéticos entre 1990 y 2017 en 195 países, señala que una de cada cinco muertes en el mundo está relacionada con la mala alimentación, sin considerar las muertes por desnutrición \cite{thelancet}. Por lo tanto, el desarrollo se sustenta en la constante necesidad y la creciente demanda de sistemas que ayuden y faciliten a una adecuada alimentación o dietoterapia para el cuidado personal.\\

Actualmente, la tecnología multimedia se abre paso en el ámbito educacional siendo un método muy explotado para el aprendizaje, llegando incluso a ser una asignatura del Grado de Magisterio en diversas universidades de España  \cite{multimedia}. \\

\begin{quote}
\textit{El estilo de aprendizaje autoaprendizaje, es más efectivo cuando reconoces que es lo más importante para ti"}
\cite{autoaprendizaje}
\end{quote}
Tras estudiar varios métodos de aprendizaje, se ha considerado que cuidar la salud a través de la alimentación no se basa exclusivamente en una dieta, sino en un estilo de vida, y el autoaprendizaje sería el mas oportuno. A un ritmo lento y constante, el usuario aprende a comer bien y no se pierde entre estrictas dietas.\\

Se pensaron en una serie de pasos imprescindibles para inculcar un nuevo estilo de vida y su relación con el autoaprendizaje:
\begin{enumerate}
\item	Identificar que se quiere cambiar.
\item	Metas específicas, realistas y constantes.
\item	Creación de un plan.
\item	Recordatorios que seguir.
\item	Mide tus avances.
\end{enumerate}
Se ha logrado que la herramienta cree una sensación de falsa simplicidad, es decir, se ha buscado hacerlo visualmente más simple, para abordar cuestiones complejas. De esta manera y como veremos a continuación, los puntos anteriores los cuales son cimientos sobre los se ha basado el autoaprendizaje que imparte la aplicación, se cumplen en los puntos siguientes, a través de las funcionalidades ya implementadas.

\begin{enumerate}
\item	Función principal, aprender y cambiar el estilo de vida a través de la dietoterapia, cumpliendo el objetivo uno de identificar que parte de tu estilo de vida deseas cambiar.
\item	Una de las partes mas complicadas e invisibles del proyecto. El programa siempre recomienda  la mejor opción, y lo realiza en base a los datos del usuario, consiguiendo que las metas sean constantes y que poco a poco se vayan volviendo mas estrictas, de forma que pase de manera imperceptible para el usuario.
\item	Respecto a la creación de un plan, aunque siempre condicionado por las recomendaciones del programa, es el usuario el que lo elige. Estas recomendaciones le ayudan a tomar la decisión correcta, ya que cada opción que decide el usuario dentro de la aplicación es registrada, y se refleja en los gráficos de calidad permitiéndole ver en todo momento la conveniencia de sus elecciones.
\item A través del Historial y los gráficos de avance, el usuario puede ser consciente de su progreso. Esta parte es muy importante para motivar al usuario y que siga adelante con el plan. Dentro de estas opciones aparecen reflejados los puntos 4 y 5.
\end{enumerate}

A lo largo de esta memoria será plausible el cómo se ha desarrollado cada parte del proyecto, tanto en el ámbito informático como en el nutricional. Detallando cada estudio, desarrollo y opción que se ha tenido a lo largo de este, para que toda decisión tomada a lo largo del periodo de desarrollo quede clara, y se entienda el por qué de cada decisión. Con este proyecto se consigue proyectar la unión de dos mundos complejos, de la manera más simple posible, haciendo alarde de que en la sencillez se oculta toda su complejidad.
