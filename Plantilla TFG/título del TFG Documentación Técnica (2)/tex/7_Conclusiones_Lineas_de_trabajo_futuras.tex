\capitulo{7}{Conclusiones y Líneas de trabajo futuras}

A continuación, se desarrollarán las diferentes conclusiones surgidas a lo largo de la creación del proyecto, a la par, de las líneas futuras de desarrollo que el proyecto a de seguir para la explotación a nivel tanto empresarial como informático.
\\
\section{Conclusiones}
A continuación, se desarrollarán la conclusión que el proyecto ha ido dejando a lo largo de su creación:
\begin{itemize}
\item	Aunque pueda pasar desapercibido, es la dificultad de inculcar, enseñar o mostrar un camino a un estilo de vida mas saludable a través de una aplicación para ordenador. En principio no se tienen ningún tipo de información, ni los recursos necesarios en cuanto a métodos pedagógicos, ni bases de datos que lo contemplen, por lo que se partió de cero, basándonos en diseños, parcialmente similares, pero sin asemejarse demasiado a lo necesitado. Resulto un reto conseguir encontrar la mejor forma de hacer que el usuario vaya aprendiendo a través de sus propias elecciones y errores.
\item	El algoritmo de recomendación, pese a todo lo aprendido a lo largo de la carrera, incluyendo entre las asignaturas, algunas donde se estudiaron los sistemas de recomendación, siempre se basaban en productos u usuarios con similitud entre ellos, dando unos pesos, y valorando cada opción para poder hacer la recomendación. Este proyecto, al nacer desde cero, no se podía realizar ninguno de esos tipos de recomendación pues era necesario una gran cantidad de registros en la base de datos de la que no se disponía, además, no se buscaba una recomendación por similitud, sino, que al usuario se le recomendará cual era la mejor opción para el dependiendo de cada momento del día y de lo que llevará comido. Esto tan trivial, trajo consigo un inmenso trabajo de investigación para poder hallar una puntuación, que serviría como ranking de comida frente a las necesidades momentáneas del usuario.
\item	Algo que parece irónico, es que, al inicio del proyecto se decidió usar un Excel para simular la base de datos, y que así fuese más sencillo el tratamiento de los datos. De este modo, no se perdería tiempo en un ámbito ajeno al objetivo principal del proyecto que era el análisis y distribución de los datos para la enseñanza de dietoterapia. Pero lo que me encontré, es que trajo más problemas que soluciones, teniendo que tener yo en cuenta la integridad de la base de datos simulada, y haciendo al final, pensar más líneas sobre como simularla lo mejor posible, en vez de crear una entera, lo cual, durante el periodo de practicas del alumno, este aprendió a manejarlas con libertad en Python presentando bastante menos problemas, que el Excel.
\item	Fue complicado la idea de desarrollar un proyecto desde cero, a lo largo de los estudios del alumno, se han realizado varios programas de distinta índole, pero siempre se partía de una base dada por el profesor y un guion a seguir. La idea de crear un proyecto, y tener que ir desarrollando y creando una serie de objetivos, fue complicado, pues muchas veces eran objetivos absurdos, imposibles, o que se desviaban totalmente del sino del proyecto.
\item	La idea de crear por primera vez un diseño, parecía sencillo, pero cumplimentar todos los detalles concretos de una buena ergonomía, para evitar que el usuario se pierda en la aplicación resulto difícil.
\item	La recursividad, como se ha tratado anteriormente en estas memorias, debido a la interfaz gráfica de Python, se trabajó con una serie de funciones recursivas que iban variando ligeramente cada vez que se las llamaba, para poder ofrecer al usuario la mejor recomendación según su elección. Esto resulto un reto mas arduo de lo que se esperaba inicialmente, siendo por un momento, un factor que puso en riesgo el correcto funcionamiento del programa.
\item	El trabajo de investigación también fue complejo y extenso, sobre todo en el ámbito mas médico-alimenticio donde existen mil fuentes de información de las cuales muy pocas son fiables. Por ello se opto por una vía mas personal como es la búsqueda de profesionales en la materia que pudiera apoyarme.
\end{itemize}

\section{Lineas de trabajo futuras}
Al ser un proyecto ambicioso, que cubre dos mundos cada vez mas conexos, existen varios métodos de expansión de la aplicación para su correcta explotación.
\subsection{Base de datos Online}
Evidentemente, era uno de los principales puntos de desarrollo si se quería seguir explotando la aplicación, la necesidad de crear una base de datos, o servidor, que te permita no solo mantener los archivos del sistema a salvos, sino que además te permite el inicio de sesión del usuario desde cualquier ordenador el cual tenga descargado. Esto crearía una mayor versatilidad a la hora de su expansión y seria uno de los puntos clave de su avance.
\subsection{Perfeccionamiento del algoritmo de recomendación}
Actualmente es un algoritmo que tiene en cuenta bastante bien las necesidades de cada usuario en cada comida, pero que no está culminado del todo. ¿Qué pasa si para mi es más conveniente un macronutriente concreto?, o ¿sí, he comido mucho un alimento? La fórmula debería tener en cuenta el LRE y la calidad de la comida en pequeñas cantidades, y los nutrientes como las grasas saturas y el azúcar para una mayor precisión.
\subsection{Aumento de la base de datos}
Obviamente se parte de una pequeña muestra, de todos los posibles menús y patología que se podrían añadir a la base de datos. No deja de ser un Start-Up de un gran desarrollo, con los ejemplos básicos para la muestra de su funcionamiento.
\subsection{Autoaprendizaje del programa}
Uno de los campos que se quedó sin desarrollar por la falta de tiempo, es de una pequeña inteligencia artificial, o variable adaptativa vinculada al usuario, la cual fuese aprendiendo para hacer que el usuario tuviera una experiencia más personal con el prolongado uso del programa.
Se pensó en una variable almacenada en la base de datos (Como solución rápida), que si en un determinado periodo de tiempo el usuario no había evolucionado en su progreso hacia la meta del peso que quería esta variase en “X” calorías para ajustarse más a las necesidades del usuario, pues todo calculo existente hasta la fecha es demasiado genérico y no funciona para todos los usuarios por igual.

\subsection{Adaptación a diferentes plataformas}
Obviamente, es una aplicación de autoaprendizaje que intenta llegar al mayor número de personas, por lo que se debería expandir a diferentes plataformas, y no exclusivamente a una aplicación de escritorio. Principalmente, a su desarrollo para SmartPhones.
\subsection{Desarrollo empresarial}
Este subapartado es meramente explicativo, y solo propone posibles métodos para una futura explotación, no esta estrechamente relacionado con el desarrollo puramente informático, sino con su posible explotación como producto.
\\
Si este proyecto quisiera verse en un ámbito empresarial para su desarrollo y explotación, habría que tener en cuenta una serie de medidas y mercados:
\begin{enumerate}
\item	Como se habló antes la expansión de la aplicación a distintas plataformas multimedia (como Smartphones).
\item	La constitución de versiones diferentes para su explotación
\begin{itemize}
\item	Free: Sustentada a través de anuncios y con una funcionalidad limitada
\item	Pro: De pago mensual y con todas las funcionalidades posibles.
\item	Enterprise: Funciones específicas, entregados a empresas dedicadas a la salud y nutrición y su posible clientela, venta por packs.
\end{itemize}
\item	Creación de un sistema de atención al cliente para dudas, y contratación de expertos, para valorar la calidad de los platos, y no llevarlo a cabo a través del algoritmo Nutriscore evitando lagunas como el que este tiene.
\item	Campaña de marketing
\item	Contratación de personal, oficina, recursos, asesoramiento, para el lanzamiento del producto al mercado.
\end{enumerate}
