\capitulo{2}{Objetivos del proyecto}

En este apartado hablaremos sobre los objetivos generales y específicos de este proyecto, aclarando las dudas sobre las intenciones de una de sus partes.
\section{Objetivos generales}
Desarrollar un Software que ponga a disposición del usuario, un sistema automatizado de planificación alimenticia, que mecanice y facilite el autoaprendizaje. De esta manera, será visible al usuario cómo influye cada decisión que toma en su día a día. Se da la misma importancia a cada comida para evitar el “por una vez no pasa nada”. \\
\\
Lograr una interfaz sencilla, apta para cualquier usuario, además de una programación simplista y modularizada para facilitar el futuro desarrollo del proyecto.
\\
\section{Objetivos Funcionales }
Los principales objetivos funcionales son:
\begin{itemize}
\item Crear una base de datos con la información necesaria para el correcto funcionamiento de la aplicación.
\item Desarrollar una aplicación de escritorio completa que proporcione:
\begin{itemize}
\item Crear nuevos usuarios, modificar la información del usuario dado de alta y añadir nuevos alimentos.
\item Visualizar a través de diferentes medios el:  progreso, historial y elecciones de un usuario concreto.
\item Elegir entre diferentes posibilidades recomendadas adecuadamente, o actualizar dichas opciones en caso de querer algo diferente.
\item Acceso constante al manual de usuario ante cualquier duda o inconveniente.
\item Calcular la calidad de un menú y sus diferentes características de manera automática.
\end{itemize}
\item Adquirir nuevos conocimientos en el ámbito dietoterapéutico.
\item Aumentar los conocimientos en el desarrollo de aplicaciones y documentos oficiales.
\end{itemize}
\section{Objetivos incrementales del desarrollo}
El proyecto fue fragmentado en pequeños y diversos objetivos específicos para conseguir llegar al objetivo final. Se valoraron objetivos complementarios para conseguir un desarrollo incremental y mantenido, haciendo que todos los objetivos específicos encajaran de manera secuencial y facilitando así su constante desarrollo, evitando al máximo posible parches que retrasaran la creación del proyecto.
\begin{itemize}
\item La idea: Lo primero de todo ha desarrollar fue la idea, había que conseguir una idea original para una herramienta nueva e innovadora, algo que cubriera alguna necesidad del mundo actual. Entonces se decidió unir los conocimientos adquiridos estos años de estudio con un tema de interés actual y en auge, así surgió la idea. Entonces se hizo única, haciendo que el usuario escogiese entre diversas opciones, y teniendo en cuenta la salud de este.
\item Dado que la idea ya estaba desarrollada, se paso a estructurar el proyecto teniendo en cuanta que tipo de personas usarían esta herramienta, se pensó en como poner a disposición del usuario la información que el programa iba a mostrarle. Se planificó una metodología de autoaprendizaje, e interfaz básica para evitar toda distracción y facilitar tanto el desarrollo como el uso.
\item Una vez aclarado todo concepto teórico sobre el desarrollo, se pasó a pensar los objetivos a nivel de desarrollo del software, y se llevó a cabo la creación y estructura de la base de datos. Se crearon dos colecciones: los alimentos y los usuarios, con sus diferentes características para el posterior tratamiento de los datos, además de una tercera que llevase el registro de la actividad del usuario.
\item Una vez pensada la mejor disposición y estructura de las bases de datos, se empezó a crear el esqueleto del programa. Se desarrolló una aplicación de línea de comandos, que de manera básica y lineal, pidiese la comprobación de usuarios y mostrara las opciones a comer.
\item	Una vez hecho el esqueleto, se empezaron a desarrollar los cálculos necesarios para redistribuir la información y llevar a cabo los cálculos de la manera más precisa posible.
\item	Ya creados los módulos y hechos los cálculos se empezó a trabajar sobre la interfaz gráfica. En este paso hubo un pequeño parche, puesto que se tuvo que adaptar el esqueleto para encajarle con las competencias de Tkinter para la interfaz gráfica.
\item	En este punto ya se había conseguido una aplicación sencilla y básica, y había que empezar a cubrir las necesidades del usuario. Era necesario que el usuario pudiera cambiar su elección, que tuviera a la vista la información de cada comida y la información sobre su día a día. Aquí se creo el tronco del programa, que se basa en actualizar todas las comidas cada vez que se actualiza los datos del actual día del cliente. Para esto se creó el módulo vista.
\item	Estando ya la aplicación funcionando y todos los algoritmos haciendo su correspondiente trabajo, tocó perfilar detalles, a simple vista de menor importancia, pero que en realidad jugaban un gran papel en el objetivo final del proyecto. Se empezó a crear el sistema de gráficos e historiales para que los usuarios pudieran aprender de sus errores, y tener un seguimiento de su avance desde el inicio del programa.
\item Cuando por fin se crearon los gráficos, y  el programa estaba operativo, se decidió dar la posibilidad al usuario de añadir nuevos menús a la aplicación. Así, se permite un mayor crecimiento y se acaba con uno de los grandes límites que presentaba el proyecto. Junto con este punto se creó la posibilidad de que un nuevo usuario se registrará en el programa.

\end{itemize}